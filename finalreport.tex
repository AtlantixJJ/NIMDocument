\documentclass{ctexart}

% if you need to pass options to natbib, use, e.g.:
% \PassOptionsToPackage{numbers, compress}{natbib}
% before loading nips_2016
%
% to avoid loading the natbib package, add option nonatbib:
% \usepackage[nonatbib]{nips_2016}

%\usepackage{nips_2016}

% to compile a camera-ready version, add the [final] option, e.g.:
\usepackage[final]{finalreport}

\usepackage[utf8]{inputenc} % allow utf-8 input
\usepackage[T1]{fontenc}    % use 8-bit T1 fonts
\usepackage{hyperref}       % hyperlinks
\usepackage{url}            % simple URL typesetting
\usepackage{booktabs}       % professional-quality tables
\usepackage{amsfonts}       % blackboard math symbols
\usepackage{nicefrac}       % compact symbols for 1/2, etc.
\usepackage{microtype}      % microtypography
\usepackage{CJKutf8}
\usepackage{titlesec}

\title{《人工神经网络》大作业最终报告}


% The \author macro works with any number of authors. There are two
% commands used to separate the names and addresses of multiple
% authors: \And and \AND.
%
% Using \And between authors leaves it to LaTeX to determine where to
% break the lines. Using \AND forces a line break at that point. So,
% if LaTeX puts 3 of 4 authors names on the first line, and the last
% on the second line, try using \AND instead of \And before the third
% author name.

\author{
  黄民烈\thanks{可利用脚注提供作者的更多信息} \\
  计算机科学与技术系 \\
  清华大学 \\
  \texttt{aihuang@tsinghua.edu.cn} \\
  %% examples of more authors
  \AND
  柯沛\\
  计算机科学与技术系 \\
  清华大学 \\
  \texttt{kepei1106@outlook.com} \\
  %% \And
  %% Coauthor \\
  %% Affiliation \\
  %% Address \\
  %% \texttt{email} \\
}

\begin{document}

\maketitle

\begin{abstract}

摘要应居中,不超过300字。

\end{abstract}


\section{引言}

本部分介绍你研究的问题,并概述解决问题的方案。


\section{相关工作}

本部分综述与你的大作业相关的文献。

\section{方法}

本节详细描述大作业的框架,注意合理使用公式、图和表格。这里给出一个公式的示例:
\begin{equation}
P(\mathbf{Y}|\mathbf{X})  =  P(Y_R, Y_L|\mathbf{X}) = P(Y_R|\mathbf{X})P(Y_L|\mathbf{X}, Y_R)
\end{equation}

\section{实验}

本节中,你需要介绍:实验的设计、数据集的使用情况以及评价方案。该部分需要同时提供量化的数值分析(可以通过图、表格来呈现)和一些具体的结果示例。


\textbf{注意}: 可以参考开题报告中给出的图表示例来描述你的实验结果。


\section{结论}

叙述你从大作业中学到了什么,以及关于大作业今后的一些想法。

\section*{参考文献}


\medskip

\small

[1] Alexander, J.A.\ \& Mozer, M.C.\ (1995) Template-based algorithms
for connectionist rule extraction. In G.\ Tesauro, D.S.\ Touretzky and
T.K.\ Leen (eds.), {\it Advances in Neural Information Processing
  Systems 7}, pp.\ 609--616. Cambridge, MA: MIT Press.

[2] Bower, J.M.\ \& Beeman, D.\ (1995) {\it The Book of GENESIS:
  Exploring Realistic Neural Models with the GEneral NEural SImulation
  System.}  New York: TELOS/Springer--Verlag.

[3] Hasselmo, M.E., Schnell, E.\ \& Barkai, E.\ (1995) Dynamics of
learning and recall at excitatory recurrent synapses and cholinergic
modulation in rat hippocampal region CA3. {\it Journal of
  Neuroscience} {\bf 15}(7):5249-5262.

\end{document}
